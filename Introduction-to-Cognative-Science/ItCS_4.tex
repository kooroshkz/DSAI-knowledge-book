\documentclass[11pt]{article}
\usepackage{amsmath}
\usepackage{graphicx}
\usepackage{hyperref}

\hypersetup{
	linktoc=all,
	hidelinks,
}

\begin{document}
	\thispagestyle{empty}
	\tableofcontents
	
	\setcounter{page}{1}
	\newpage
	\section{Language and micro-worlds}
	Eliza was the first ever chat-bot, created at MIT to mimic psychotherapists.\\
	When people interacted with Eliza sensed that it was able to understand them due to its reaction to keywords despite the simplicity of the tricks used to program it. This led to the anthropomorphizition of the computer code.\\
	Parry is a simulated patient with paranoid schizophrenia made for a modified Turing test, it was often inconsistent and made meaningless sentences making it seem more realistic. The majority of psychiatrists asked to identify whether it was human or machine failed.\\
	SHRDLU is a micro-world that is a specific domain that limits objects of reality to make interactions easier to understand by robotics, it was created as a method to simulate understanding language. It had a syntactic analysis where it identifies nouns, verbs, etc\dots in order to parse the sentences. It also had semantic analysis where it assigned meaning to individual words. Lastly it also had a basic understanding of the world and how meanings interact.\\
	Using language is algorithmic in nature. Then why do we need world knowledge?\\
	We cannot use only the meaning of individual words to remove ambiguity of the sentence, we need an understanding of how those meanings interact in order to have a clear understanding of the sentence meaning.\\
	Cognitive science sees humans as algorithmic processors.\\
	
	\section{Visual imagery and imagination}
	Our imagination is bound to the same cognitive and physical limitations of our regular perception.\\
	A spatial or depictive representation is the visual image of your visualization, as if it is there in the real world.\\
	A propositional representation is descriptive of the image but not the image itself, like a text description.
	\newpage
	\subsection{The problem of vision}
	Marr was a neuroscientist who said that there are three levels of analysis, the computational level, the algorithmic level, and the implementation level.\\
	He gave the example of object recognition, he gave a visual input and received oral output. Marr says that we start with an image on the retina, then we have three stages of processing, the first is primal sketch where you find light and dark transition which connect to form blobs and edges.\\
	There are different types of constancy which we cannot control, like the brightness, shape, and size consistencies.\\
	Our brain corrects visual information based on our understandings.\\
	How do we recognize objects? We have mental representations of classes known as concepts, concepts are multi-modal. All concepts belong to categories, categories have some broad attributes.\\
	There are different way to categorize, by definition, by prototype, by exemplars, and by components.\\
	A prototype is the idealization of a category and doesn't need to be an actual existing instance.\\
	The resemblance of an instance to a prototype is its typicality, which can be measured using sentence verification tasks.\\
	
	\section{Representational hierarchies}
	Experts categorize on a more specific level if you show them instances in their expertise field.\\
	
\end{document}
